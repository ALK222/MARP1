\documentclass[6pt]{AiTex}




\title{Memoria activad opcional}
\author{A.L.K.}
\date{Diciembre 2022}

\begin{document}
%\datos{facultad}{universidad}{grado}{asignatura}{subtitulo}{autor}{curso}
\datos{Informática}{Universidad Complutense de Madrid}{Ingeniería informática}{Métodos Algorítmicos en Resolución de Problemas}{Algoritmo de Prim y montículo de Williams}{Alejandro Barrachina Argudo}{2022-2023}
% \portadaApuntes
% \pagestyle{empty}
% \tableofcontents
% \pagestyle{empty}
\justify


\section*{Introducción}

En este documento se presenta la explicación de la practica opcional de MAR1, apartado 8, realizada por \autor.

\begin{multicols}{2}

    \section{Algoritmo de Prim}
% \section{Estructura de archivos y ejecución}
\subsection{Estructura de archivos}
{
    \small
    La carpeta está estructurada de la siguiente manera:

    \begin{itemize}
        \item\textbf{bin:} Carpeta que contiene el archivo final de compilación.
        \item\textbf{graph:} Carpeta que contiene los archivos de la biblioteca de grafos y casos de prueba:
        \begin{itemize}
            \item\textbf{graph.h y graph.cpp:} archivos con el código de implementación de grafos mediante listas de adyacencia y el propio algoritmo de Prim.
            \item\textbf{dummygraphs.h:} cabecera que introduce los grafos al programa.
            \item\textbf{test:} carpeta que contiene un generador de grafos aleatorio y los casos de prueba generados
        \end{itemize}
        \item\textbf{obj:} carpeta para la compilación de la librería de grafos.
        \item\textbf{main.cpp}: archivo fuente que corre los casos de prueba.
        \item\textbf{makefile}: archivo make para facilitar la compilación de la práctica.
    \end{itemize}

    La implementación del grafo usada es una implementación similar a la vista en clase utilizando un vector de aristas para cada nodo, sustituyendo en esta lista el identificador del nodo por un par identificador-peso.
    Esto nos permitirá encontrar el camino menos costoso entre todos los nodos del grafo.





    Los gráficos de las pruebas se pueden ver en la figura \ref{fig:prim-plot}~(\nameref{fig:prim-plot}). Estas pruebas incluyen la carga de un grafo desde fichero con todo lo que ello conlleva, pero no la generación del grafo como tal.




    Los tiempos observados se pueden asimilar a las de una función $(a+v)log(v)$, siendo a las aristas y v los vértices, que es lo que se pedía en el enunciado.
    A mayor numero de aristas en el grafo, mayor será el coste.
}

    \vfill
    \hfill

    \section{Montículo de Williams con Decrecer clave}
 {\small
  \subsection{Estructura de archivos}
  La carpeta está estructurada de la siguiente manera:
  \begin{itemize}
      \item\textbf{bin:} Carpeta que contiene el archivo final de compilación.
      \item\textbf{williams-heap:} Carpeta que contiene los archivos de la biblioteca de montículos y casos de prueba:
      \begin{itemize}
          \item\textbf{williams-heap.h y williams-heap.cpp:} archivos con el código de implementación de montículos mediante listas de adyacencia y el propio algoritmo de Prim.
          \item\textbf{dummyheaps.h:} cabecera que introduce los montículos al programa.
          \item\textbf{test:} carpeta que contiene un generador de montículos aleatorio y los casos de prueba generados
      \end{itemize}
      \item\textbf{obj:} carpeta para la compilación de la librería de montículos.
      \item\textbf{main.cpp}: archivo fuente que corre los casos de prueba.
      \item\textbf{makefile}: archivo make para facilitar la compilación de la práctica.
  \end{itemize}

  La implementación del montículo usada es la vista en los apuntes de clase, traduciendo los métodos de Daphny a C++.


  Los gráficos de las pruebas se pueden ver en la figura \ref{fig:decrease-key-plot}~(\nameref{fig:decrease-key-plot}). Estas pruebas incluyen la carga de un grafo desde fichero con todo lo que ello conlleva, pero no la generación del grafo como tal.

  Los tiempos observados se pueden asimilar a las de una función $log(n)$, siendo n el número de elementos del montículo, que es lo que se pedía en el enunciado. Podemos ver picos y valles en las gráficas dado que los grafos de un mismo caso pueden tener un número muy distinto de aristas dada la naturaleza aleatoria de su generación.
 }

\end{multicols}

\newpage

\section{Ejecución}

Distinguiremos \textbf{carpeta raíz} y \textbf{ejecutable} según cada apartado:
\begin{itemize}
    \item Para el algoritmo de prim:
          \begin{itemize}
              \item\textbf{Carpeta raíz:} src/prim-algo
              \item\textbf{Ejecutable:} Prim-algo
          \end{itemize}
    \item Para el montículo de Williams:
          \begin{itemize}
              \item\textbf{Carpeta raíz:} src/decrease-key
              \item\textbf{Ejecutable:} Williams-decrease
          \end{itemize}
\end{itemize}

Si queremos probar un pequeño (muy pequeño) caso de prueba, podemos ejecutar el siguiente comando sobre la carpeta raíz:
\begin{lstlisting}[style=custombash]
    make clean all
    ./bin/${ejecutable}.out
    \end{lstlisting}

Si queremos ejecutar una prueba concreta, sobre la carpeta anteriormente mencionada ejecutaremos:
\begin{lstlisting}[style=custombash]
    make clean prueba
    ./bin/${ejecutable}-prueba.out RUTA_AL_ARCHIVO_DE_PRUEBA
    \end{lstlisting}

En ambos casos podemos añadir al final del comando make \lstinline[style=custombash]{CFLAGS:=$(CFLAGS)"-D __DISPLAY"} para mostrar el resultado de la operación.

Si queremos ejecutar la batería de pruebas proporcionadas junto a esta entrega solo hay que ir a la carpeta anteriormente ejecutada y lanzar \lstinline[style=custombash]{./runner.sh}.



\section{Figuras}

\begin{figure}[H]
    \includegraphics[width = 0.8\textwidth]{./images/average-prim-plot.png}
    \caption{Gráfica de tiempo del algoritmo de Prim para distintas cargas de trabajo.}
    \label{fig:prim-plot}
\end{figure}

\begin{figure}[H]
    \includegraphics[width = 0.8\textwidth]{./images/prim-plot.png}
    \caption{Distribución de los tiempos observados en las distintas muestras del algoritmo de Prim para distintas cargas de trabajo.}
    \label{fig:prim-plot}
\end{figure}


\begin{figure}[H]
    \includegraphics[width = 0.75\textwidth]{./images/average-decrease-key-plot.png}
    \caption{Gráfica de tiempo de Decrecer clave para varios tamaños de montículo.}
    \label{fig:decrease-key-plot}
\end{figure}

\begin{figure}[H]
    \includegraphics[width = 0.8\textwidth]{./images/decrease-key-plot.png}
    \caption{Distribución de los tiempos observados en las distintas muestras de Decrecer clave para distintas cargas de trabajo.}
    \label{fig:prim-plot}
\end{figure}




\end{document}
